% Settings for debugging
%\tracinggroups=1
%\tracingnesting=2

% standalone package, "simplifies" compilations of multi-source document
%\usepackage{standalone}

% custom packages
\usepackage{sty/er-abbrev}
\usepackage{sty/er-math}
\usepackage{sty/er-paper}
\usepackage{sty/mk-outline}
\usepackage{sty/mk-chem-notation}
\usepackage{sty/mk-ffpilot-notation}
\usepackage{sty/mk-stats-notation}

% some geometry setting stuff. Useful if not done elsewhere
%\usepackage[top=0.85in,left=0.75in,footskip=0.75in]{geometry}

% for bib file and bibliography support
%\usepackage[numbers,sort&compress]{natbib}

%\usepackage{amsmath,amssymb}
%\usepackage{mathtools}

% Use Unicode characters when possible
%\usepackage[utf8]{inputenc}

% for acronyms
%\usepackage[nolist]{acronym}

% line numbers
\usepackage[right]{lineno}

% don't remember what this is for
\usepackage{calc}

% Load any custom macros or abbreviations from abbreviations.tex
% Abbreviations and command definitions.
\defineAbbreviation{CA}{cellular automata}
\defineAbbreviation{CET}{cryoelectron tomography}
\defineAbbreviation{CME}{chemical master equation}
\defineAbbreviation{CPU}{central processing unit}
\defineAbbreviation{CUDA}{Compute Unified Device Architecture}
\defineAbbreviation{DGL}{differential gene loss}
\defineAbbreviation{DS}{direct sampling}
\defineAbbreviation{ds-protein}{domain specific ribosomal protein}
\defineAbbreviationPlural{ds-protein}{ds-proteins}{domain specific ribosomal proteins}
\defineAbbreviation{EF-Tu}{elongation factor Tu}
\defineAbbreviation{EN}{extrinsic noise}
\defineAbbreviation{EP}{evolutionary profile}
\defineAbbreviation{ERM}{equal-rates Markov}
\defineAbbreviation{ES}{enhanced sampling}
\defineAbbreviation{FBA}{flux balance analysis}
\defineAbbreviation{FFPilot}{forward flux pilot sampling}
\defineAbbreviation{FPE}{Fokker-Planck equation}
\defineAbbreviation{GFP}{green fluorescent protein}
\defineAbbreviation{GPU}{graphics processing unit}
\defineAbbreviationPlural{GPU}{GPUs}{graphics processing units}
\defineAbbreviation{GRF}{gene regulation function}
\defineAbbreviation{GMP}{Gillespie multi-particle}
\defineAbbreviation{GTS}{genetic toggle switch}
\defineAbbreviation{HDFS}{Hadoop distributed file system}
\defineAbbreviation{HGT}{horizontal gene transfer}
\defineAbbreviation{HPC}{high performance computing}
\defineAbbreviation{IN}{intrinsic noise}
\defineAbbreviation{indel}{insertion or deletion}
\defineAbbreviationPlural{indel}{indels}{insertions or deletions}
\defineAbbreviation{IPTG}{isopropyl $\beta$-D-1-thiogalactopyranoside}
\defineAbbreviation{KL}{Kullback--Leibler}
\defineAbbreviation{LacI}{{\it lac} repressor}
\defineAbbreviation{LacY}{lactose permease}
\defineAbbreviation{LMES}{Lattice Microbes ES} 
\defineAbbreviation{LSU}{large subunit}
\defineAbbreviation{ME-EP}{maximum-entropy evolutionary profile}
\defineAbbreviation{MD}{molecular dynamics}
\defineAbbreviation{MFPT}{mean first passage time}
\defineAbbreviation{ML}{maximum-likelihood}
\defineAbbreviation{MPD}{mulitparticle diffusion}
\defineAbbreviation{MPD-RDME}{mulitparticle-diffusion RDME}
\defineAbbreviation{MR-RDME}{multiresolution RDME}
\defineAbbreviation{mRNA}{messenger RNA}
\defineAbbreviation{MSA}{multiple sequence alignment}
\defineAbbreviation{MSD}{mean square displacement}
\defineAbbreviation{MST}{mean switching time}
\defineAbbreviationPlural{MST}{MSTs}{mean switching times}
\defineAbbreviation{nm}{nanometers}
\defineAbbreviation{ns}{nanoseconds}
\defineAbbreviation{ODE}{ordinary differential equation}
\defineAbbreviationPlural{ODE}{ODEs}{ordinary differential equations}
\defineAbbreviation{OU}{Ornstein-Uhlenbeck}
\defineAbbreviation{PDE}{partial differential equation}
\defineAbbreviationPlural{PDE}{PDEs}{partial differential equations}
\defineAbbreviation{PDF}{probability density function}
\defineAbbreviationPlural{PDF}{PDFs}{probability density functions}
\defineAbbreviation{PFB}{positive feedback}
\defineAbbreviation{QBIO}{quantitative biology}
\defineAbbreviation{RBS}{ribosomal binding site}
\defineAbbreviation{RMSD}{root-mean-square deviation}
\defineAbbreviation{RNAP}{RNA polymerase}
\defineAbbreviation{REU}{Research Experiences for Undergraduates}
\defineAbbreviation{RDME}{reaction-diffusion master equation}
\defineAbbreviation{rRNA}{ribosomal RNA}
\defineAbbreviation{r-protein}{ribosomal protein}
\defineAbbreviationPlural{r-protein}{r-proteins}{ribosomal proteins}
\defineAbbreviation{SCT}{single-cell trap}
\defineAbbreviation{SSA}{stochastic simulation algorithm}
\defineAbbreviation{SSU}{small subunit}
\defineAbbreviation{SI}{supporting information}
\defineAbbreviation{SOQ}{Summer of QBIO}
\defineAbbreviation{SRG}{self-regulating gene}
\defineAbbreviation{SRP}{signal recognition particle}
\defineAbbreviation{TMG}{thiomethyl-$\beta$-D-galactoside}
\defineAbbreviation{TRN}{transcriptional regulatory network}
\defineAbbreviation{TOC}{table of contents}
\defineAbbreviation{UPT}{universal phylogenetic tree}
\defineAbbreviation{VM}{virtual machine}

% Species abbreviations.
\defineAbbreviationStrict{Ametal}{\textit{A. metalliredigens}}{\textit{Alkaliphilus metalliredigens}}
\defineAbbreviationStrict{Aorem}{\textit{A. oremlandii}}{\textit{Alkaliphilus oremlandii}}
\defineAbbreviationStrict{Bsubt}{\textit{B. subtilis}}{\textit{Bacillus subtilis}}
\defineAbbreviationStrict{Cacet}{\textit{C. acetobutylicum}}{\textit{Clostridium acetobutylicum}}
\defineAbbreviationStrict{Celegans}{\textit{C. elegans}}{\textit{Caenorhabditis elegans}}
\defineAbbreviationStrict{Cnovyi}{\textit{C. novyi}}{\textit{Clostridium novyi}}
\defineAbbreviationStrict{Dradio}{\textit{D. radiodurans}}{\textit{Deinococcus radiodurans}}
\defineAbbreviationStrict{Ecoli}{\textit{E. coli}}{\textit{Escherichia coli}}
\defineAbbreviationStrict{Fmagna}{\textit{F. magna}}{\textit{Finegoldia magna}}
\defineAbbreviationStrict{Hmaris}{\textit{H. marismortui}}{\textit{Haloarcula marismortui}}
\defineAbbreviationStrict{Lborg}{\textit{L. borgpetersenii}}{\textit{Leptospira borgpetersenii}}
\defineAbbreviationStrict{Mflag}{\textit{M. flagellatus}}{\textit{Methylobacillus flagellatus}}
\defineAbbreviationStrict{Mtuber}{\textit{M. tuberculosis}}{\textit{Mycobacterium tuberculosis}}
\defineAbbreviationStrict{Pingr}{\textit{P. ingrahamii}}{\textit{Psychromonas ingrahamii}}
\defineAbbreviationStrict{Saren}{\textit{S.  arenicola}}{\textit{Salinispora  arenicola}}
\defineAbbreviationStrict{Scerev}{\textit{S. cerevisiae}}{\textit{Saccharomyces cerevisiae}}
\defineAbbreviationStrict{Scoel}{\textit{S. coelicolor}}{\textit{Streptomyces coelicolor}}
\defineAbbreviationStrict{Ssolf}{\textit{S. solfataricus}}{\textit{Sulfolobus solfataricus}}
\defineAbbreviationStrict{Strop}{\textit{S. tropica}}{\textit{Salinispora tropica}}
\defineAbbreviationStrict{Ttherm}{\textit{T. thermophilus}}{\textit{Thermus thermophilus}}

% Grammar shortcuts.
\newcommand{\ie}{\textit{i.e.}\xspace}
\newcommand{\eg}{\textit{e.g.}\xspace}
\newcommand{\etal}{\textit{et al.}\xspace}

% Biology shortcuts.
\newcommand{\insitu}{\textit{in situ}\xspace}
\newcommand{\invivo}{\textit{in vivo}\xspace}
\newcommand{\invitro}{\textit{in vitro}\xspace}
\newcommand{\insilico}{\textit{in silico}\xspace}

% RDME definitions.
\newcommand{\xv}{{\vec{x}}}
\newcommand{\sv}{\vec{S}}
\newcommand{\rv}{\vec{r}}
\newcommand{\drv}{\vec{dr}}
\newcommand{\Dop}{{\mathcal D}}
\newcommand{\Rop}{{\mathcal R}}

% Unit shortcuts.
\newcommand{\Dunits}{$\mathrm{\mathsf{\mu m^2/s}}$\xspace}
\newcommand{\pMps}{$\mathrm{\mathsf{M^{-1} s^{-1}}}$\xspace}
\newcommand{\ps}{$\mathrm{\mathsf{s^{-1}}}$\xspace}
\newcommand{\uM}{$\mathrm{\mathsf{{\mu}M}}$\xspace}
\newcommand{\um}{$\mathrm{\mathsf{{\mu}m}}$\xspace}
\newcommand{\umcube}{$\mathrm{\mathsf{{\mu}m^3}}$\xspace}
\newcommand{\us}{$\mathrm{\mathsf{{\mu}s}}$\xspace}
\newcommand{\mmps}{$\mathrm{\mathsf{m^{2}\,s^{-1}}}$\xspace}
\newcommand{\umumps}{$\mathrm{\mathsf{{\mu}m^{2}\,s^{-1}}}$\xspace}

% Math shortcuts.
\newcommand{\nomath}[1]{$\mathrm{\mathsf{#1}}$\xspace}

% Custom macros.
\newcommand{\ROOT}{\code{\$FILES\_ROOT}\xspace}



% don't split long footnotes across several pages
\interfootnotelinepenalty=10000

%% use symbols instead of numbers for footnotes, so they don't conflict with bibtex citations
%\renewcommand*{\thefootnote}{\fnsymbol{footnote}}
%\usepackage{perpage}
%\MakePerPage{footnote}
%
%% alias for helping to squeeze footnotes into nicely formatted tables
%\newcommand{\fnm}[1]{\footnotemark[#1]}

\usepackage[export]{adjustbox}[2011/08/13]

%%%% DEPRECATED STUFF %%%%

%% Use nameref to cite supporting information files (see Supporting Information section for more info)
%\usepackage{nameref,hyperref}
%
%% line numbers
%\usepackage[right]{lineno}
%
%% ligatures disabled
%\usepackage{microtype}
%\DisableLigatures[f]{encoding = *, family = * }
%
%% color can be used to apply background shading to table cells only
%\usepackage[table]{xcolor}
%
%% array package and thick rules for tables
%\usepackage{array}

% markdown option 1 (including the enumitem stuff)
%\usepackage{markdown}
%\markdownSetup{
%    hashEnumerators,
%    definitionLists,
%    footnotes,
%    hybrid,
%    smartEllipses,
%    renderers = {
%        emphasis = _#1,
%    },
%}

%\usepackage{enumitem}
%\setlistdepth{20}
%\renewlist{compactitem}{itemize}{20}
%\setlist[compactitem]{label=--}

% markdown option 2
%\usepackage[document]{latexmarkdn}

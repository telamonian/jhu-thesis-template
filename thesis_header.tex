% Settings for debugging
%\tracinggroups=1
%\tracingnesting=2

% standalone package, "simplifies" compilations of multi-source document
%\usepackage{standalone}

% custom packages
\usepackage{sty/er-abbrev}
\usepackage{sty/er-math}
\usepackage{sty/er-paper}
\usepackage{sty/mk-outline}
\usepackage{sty/mk-chem-notation}
\usepackage{sty/mk-ffpilot-notation}
\usepackage{sty/mk-stats-notation}

% some geometry setting stuff. Useful if not done elsewhere
%\usepackage[top=0.85in,left=0.75in,footskip=0.75in]{geometry}

% for bib file and bibliography support
%\usepackage[numbers,sort&compress]{natbib}

%\usepackage{amsmath,amssymb}
%\usepackage{mathtools}

% Use Unicode characters when possible
%\usepackage[utf8]{inputenc}

% for acronyms
%\usepackage[nolist]{acronym}

% line numbers
\usepackage[right]{lineno}

% don't remember what this is for
\usepackage{calc}

% Load any custom macros or abbreviations from abbreviations.tex
% Abbreviations and command definitions.
\defineAbbreviation{ES}{enhanced sampling}
\defineAbbreviation{DS}{direct sampling}
\defineAbbreviation{FFPilot}{forward flux pilot sampling}
\defineAbbreviation{FFS}{forward flux sampling}

\defineAbbreviationStrict{MFPT}{$\mfpttrue$}{mean first passage time ($\mfpttrue$)}

\defineAbbreviation{REM}{rare event model}
\defineAbbreviation{SRG}{self regulating gene}
\defineAbbreviation{GTS}{genetic toggle switch}

\defineAbbreviation{PBD}{Poisson binomial distribution}

\defineAbbreviationStrict{AA}{$\mathcal{A}$}{high A state}
\defineAbbreviationStrict{BB}{$\mathcal{B}$}{high B state}

%\begin{acronym}
%    \acro{REM}[REM]{rare event model}
%\end{acronym}

%\defineAbbreviationPlural{r-protein}{r-proteins}{ribosomal proteins}
%\defineAbbreviationStrict{Ametal}{\textit{A. metalliredigens}}{\textit{Alkaliphilus metalliredigens}}

% Grammar shortcuts.
\newcommand{\ie}{\textit{i.e.}\xspace}
\newcommand{\eg}{\textit{e.g.}\xspace}
\newcommand{\etal}{\textit{et al.}\xspace}
\newcommand{\apriori}{\textit{a priori}\xspace}

% Biology shortcuts.
\newcommand{\insitu}{\textit{in situ}\xspace}
\newcommand{\invivo}{\textit{in vivo}\xspace}
\newcommand{\invitro}{\textit{in vitro}\xspace}
\newcommand{\insilico}{\textit{in silico}\xspace}

% Unit shortcuts.
\newcommand{\Dunits}{$\mathrm{\mathsf{\mu m^2/s}}$}
\newcommand{\pMps}{$\mathrm{\mathsf{M^{-1} s^{-1}}}$}
\newcommand{\ps}{$\mathrm{\mathsf{s^{-1}}}$}
\newcommand{\uM}{$\mathrm{\mathsf{{\mu}M}}$}
\newcommand{\um}{$\mathrm{\mathsf{{\mu}m}}$}
\newcommand{\us}{$\mathrm{\mathsf{{\mu}s}}$}
\newcommand{\mmps}{$\mathrm{\mathsf{m^{2}\,s^{-1}}}$}



% don't split long footnotes across several pages
\interfootnotelinepenalty=10000

%% use symbols instead of numbers for footnotes, so they don't conflict with bibtex citations
%\renewcommand*{\thefootnote}{\fnsymbol{footnote}}
%\usepackage{perpage}
%\MakePerPage{footnote}
%
%% alias for helping to squeeze footnotes into nicely formatted tables
%\newcommand{\fnm}[1]{\footnotemark[#1]}

\usepackage[export]{adjustbox}[2011/08/13]

%%%% DEPRECATED STUFF %%%%

%% Use nameref to cite supporting information files (see Supporting Information section for more info)
%\usepackage{nameref,hyperref}
%
%% line numbers
%\usepackage[right]{lineno}
%
%% ligatures disabled
%\usepackage{microtype}
%\DisableLigatures[f]{encoding = *, family = * }
%
%% color can be used to apply background shading to table cells only
%\usepackage[table]{xcolor}
%
%% array package and thick rules for tables
%\usepackage{array}

% markdown option 1 (including the enumitem stuff)
%\usepackage{markdown}
%\markdownSetup{
%    hashEnumerators,
%    definitionLists,
%    footnotes,
%    hybrid,
%    smartEllipses,
%    renderers = {
%        emphasis = _#1,
%    },
%}

%\usepackage{enumitem}
%\setlistdepth{20}
%\renewlist{compactitem}{itemize}{20}
%\setlist[compactitem]{label=--}

% markdown option 2
%\usepackage[document]{latexmarkdn}
